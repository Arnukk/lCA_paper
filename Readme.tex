LINKS
----------------------------------
https://www.lcacommons.gov/nrel/
http://www.eiolca.net/cgi-bin/dft/use.pl
http://resilience.eng.ohio-state.edu/ecolca-cv/
-----------------------------------------------


What’s the difference?

In these times of climate change concern, individuals and organizations alike are eager for measurable criteria to compare the impact of products and services on global warming. The notions of 'Life Cycle Assessment', 'Carbon Footprint', and 'Ecological Footprint' often appear in the media, but their exact meaning and the differences between them are rarely explained or widely understood.

Calculating all the big environmental impact categories

Life Cycle Assessment (LCA) is the broadest indicator and an internationally standardized method (ISO 14040 and ISO 14044). It not only evaluates the impact on climate change, but also other impact categories such as acidification potential, eutrophication potential, ozone depletion potential, and ground level ozone creation. For each of these impact categories, the product or system is evaluated over its complete life span, from the extraction of raw material and manufacturing, to the use of the product by final consumers and end-of-life processes like recycling, energy recovery, and ultimate waste disposal.

The ISO standards provide robust and practice-proven requirements for performing transparent LCA calculations. Moreover, one can make use of extensive databases containing life cycle profiles of many goods and services, as well as many of the underlying materials, energy resources, transport systems, etc. Nevertheless, LCA calculations remain very complex and should therefore be applied only by professionals and preferably to a specific unit or application, such as a washing machine or a car tire.

Calculating the impact on climate change

A Carbon Footprint, also called Carbon Profile, is an LCA with the analysis limited to emissions that have an effect on climate change (carbon dioxide, methane, etc.). This limitation makes it easier to apply the calculation to integrated systems, such as an entire house or automobile.

Comparing with the amount of land needed

The Ecological Footprint of an activity tries to measure its consumption of natural resources and the amount of biologically productive land and sea needed to regenerate those resources and to absorb and render harmless the waste that is produced. While this measure is being increasingly used, it is not a scientific standard and is widely criticized. The translation of every impact into land and sea areas is neither self-evident nor is it easily established. Consequently, various methodologies are in use that sometimes give conflicting results. The calcultions also largely depend on the prevailing technologies, meaning that their results are evolving with the state of technology. Nevetheless, being fully aware of its limitations, Ecological Footprint calculations can sometimes be used to establish a first rough approximation of the ecological impact of a product or a system.




Why LCA approach and not other one ?

\begin{comment}
Environmental Systems Analysis (ESA) tools (see Table \ref{esa_tools}) are widely used to study how different products and services impact the environment throughout their life-time. The choice of which ESA tool to use depends on the decision makers. However, the ESA tool selected must answer question like:
\begin{enumerate}
\item{Is the tool procedural or analytical?}
\item{What type of impact study is considered?}
\item{What is the objective of the study?}
\item{Is the tool for descriptive or change-oriented studies?}
\end{enumerate}

Amongst the different ESA tools outlined in Table \ref{esa_tools}, LCA is the most preferred. 


\begin{table}[htbp]
\caption{Environment Systems Analysis Tools\cite{Moberg_environmentalsystems}}
\begin{center}
\begin{tabular}{|l|l|}
\hline
\textbf{Tool} & \textbf{Decision Support Type} \\ \hline
Cost‐Benefit Analysis, CBA & Analytical \\ \hline
Direct Material Consumption, DMC & Analytical \\ \hline
Direct Material Input, DMI & Analytical \\ \hline
Ecological Footprint, EF & Analytical \\ \hline
Environmental Impact Assessment, EIA & Procedural \\ \hline
Economic Valuation  & Evaluation \\ \hline
Environmental Management System, EMS & Procedural \\ \hline
Energy analysis, En & Analytical \\ \hline
Futures studies  & Analytical \& Procedural \\ \hline
Input‐Output Analysis, IOA & Analytical \\ \hline
Impact Pathway Approach, IPA & Analytical \\ \hline
Life Cycle Assessment, LCA & Analytical \\ \hline
Life Cycle Costing, LCC & Analytical \\ \hline
Multiple Attribute Analysis, MAA & Analytical \\ \hline
Material Intensity Per Unit Service, MIPS & Analytical \\ \hline
Material Flow Analysis, MFA & Analytical \\ \hline
Risk Assessment, RA & Analytical \\ \hline
Strategic Environmental Assessment, SEA & Procedural \\ \hline
System of Economic \& Environmental Accounts, SEEA & Analytical \\ \hline
Substance Flow Analysis, SFA & Analytical \\ \hline
Surveys  & Evaluation \\ \hline
Total Material Requirement, TMR & Analytical \\ \hline
\end{tabular}
\end{center}
\label{esa_tools}
\end{table}
\end{comment}





Discussion of related work, considerations and indetified weaknesses\\


Furthermore, \cite{gard2002digital} even come to the conclusion that when comparing digital and traditional library systems with respects to the energy consumption, one can not insist the superiority of any of the systems and that the results of LCA are largely dependant on the number of model input parametrs. What is more, in no study conducted (to our best knowledge) the process of the development of specific applications for the ICT service was accounted, which, in some cases, may lead to deviation from representing the actual eco impact of the ICT product.



\begin{itemize}
\item {\em PBC system}. The PBC system is the "system in which manual exchange of paper based business cards which is achieved by lengthy manufacturing process" compared to DBC. This process  paper based business cards. The term PBC system refers to the whole system in which the cards are manufactured while PBC refers to the product itself.


\item {\em DBC system} Digital-based business Card(DBC). The DBC system is " the system in which digital exchange of business cards occurs between two or more smart phone over the internet". The DBC software or application or Infold in this paper refers to the mobile application that facilitates the exchange of business cards between two or more people. Similarly the term DBC system refers to the whole system in which the virtual cards are formed while DBC refers to the product itself i.e.  a digital card.
\end{itemize} 